% --------------------------------------------------------------
% This is all preamble stuff that you don't have to worry about.
% Head down to where it says "Start here"
% --------------------------------------------------------------
 
\documentclass[11pt]{article}
 
\usepackage[margin=1in]{geometry} 
\usepackage{amsmath,amsthm,amssymb}
 
\newcommand{\N}{\mathbb{N}}
\newcommand{\Z}{\mathbb{Z}}
 
\newenvironment{theorem}[2][Theorem]{\begin{trivlist}
\item[\hskip \labelsep {\bfseries #1}\hskip \labelsep {\bfseries #2.}]}{\end{trivlist}}
\newenvironment{lemma}[2][Lemma]{\begin{trivlist}
\item[\hskip \labelsep {\bfseries #1}\hskip \labelsep {\bfseries #2.}]}{\end{trivlist}}
\newenvironment{exercise}[2][Exercise]{\begin{trivlist}
\item[\hskip \labelsep {\bfseries #1}\hskip \labelsep {\bfseries #2.}]}{\end{trivlist}}
\newenvironment{problem}[2][Problem]{\begin{trivlist}
\item[\hskip \labelsep {\bfseries #1}\hskip \labelsep {\bfseries #2.}]}{\end{trivlist}}
\newenvironment{question}[2][Question]{\begin{trivlist}
\item[\hskip \labelsep {\bfseries #1}\hskip \labelsep {\bfseries #2.}]}{\end{trivlist}}
\newenvironment{corollary}[2][Corollary]{\begin{trivlist}
\item[\hskip \labelsep {\bfseries #1}\hskip \labelsep {\bfseries #2.}]}{\end{trivlist}}

\newenvironment{solution}{\begin{proof}[Solution]}{\end{proof}}
 
\begin{document}
 
% --------------------------------------------------------------
%                         Start here
% --------------------------------------------------------------
 
\title{Problem Set 1}
\author{Evan Hataishi\\ %replace with your name
ICS 621: Analysis of Algorithms\\
Prof. N. Sitchinava}
\date{September 6, 2019}

\maketitle

\section{Dynamic Arrays (40 pts)}

\textit{Using the potential method, prove that insertion into the dynamic array has amortized cost of O(1).}\\

Let $D_i$ be a dynamic array, $n$ be the number of items in $D_i$, and $s$ be the size of $D_i$ after $i$ insertions. At any given time, $D_i$ will have $n - \frac{s}{2}$ more items since the most recent resize.  Each of these item's insertions must add enough "potential" to "pay" for a future resize. The potential must be enough to copy each item and a counterpart (in the first half of $D_i$). Thus, we can define our potential function to be
\begin{align*}
    \Phi (D_i) & = 2(n - \frac{s}{2}),\\
    & = 2n - s.
\end{align*}
% must show that this is always >= D_0
$D_0$ is initially an array of size 0 with no elements. Items are only inserted and never removed, so the number of items $n$ is guaranteed to be at least $\frac{s}{2}$. Therefore, this potential function is always greater than or equal to zero, and the amortized cost is an upper bound on the actual cost:
\begin{equation*}
 \Phi (D_i) \geq  \Phi (D_0). \quad \checkmark
\end{equation*}
Suppose at the $i$-th insertion, the array is already full. A new array of size $2s$ is allocated, all $n$ elements are copied over, and then the new element is finally inserted. $n$ has increased by 1, and the array size has doubled. We obtain potential functions
\begin{equation*}
    \Phi (D_i) = 2(n + 1) - 2s, \Phi (D_{i-1}) = 2n - s.
\end{equation*}
The actual cost $c_i$ is the sum of the cost to copy $s$ elements and insert the new element i.e. $c_i=s + 1$. By equation (17.2), the amortized cost of this insertion is
\begin{align*}
    \hat{c}_i & = c_i + \Phi (D_i) - \Phi (D_{i-1}),\\
    & = (s + 1) + (2(n + 1) - 2s) - (2n - s),\\
    & = 3,\\
    & = O(1).
\end{align*}
In the simpler and cheaper case, where the $i$-th insertion is into a non-full array, we only need to insert without resizing. The potential functions become
\begin{equation*}
    \Phi (D_i) = 2(n + 1) - s, \Phi (D_{i-1}) = 2n - s.
\end{equation*}
The actual cost $c_i$ is simply the cost of inserting a new item i.e. $c_i=1$. By equation (17.2), the amortized cost of this insertion is
\begin{align*}
    \hat{c}_i & = c_i + \Phi (D_i) - \Phi (D_{i-1}),\\
    & = 1 + (2(n + 1) - s) - (2n - s),\\
    & = 3,\\
    & = O(1).
\end{align*}
As we can see, both the cheap and expensive cases for insertion yield an amortized cost of O(1).

\section{Binomial Trees (60 pts)}

\textit{Prove that for the binomial tree $B_k$,}
\begin{enumerate}
    \item there are $2^k$ nodes\\
    \textbf{Base Case:} If $k=0$, $2^0=1$ node. $B_0$ is a binomial tree with only 1 node, so $k=1$ holds. \\
    \textbf{Inductive Hypothesis:} For any $k \geq 0$, a binomial tree $B_k$ has $2^k$ nodes.\\
    \textbf{Induction Step:} Prove that binomial tree $B_{k+1}$ has $2^{k+1}$ nodes.\\
    $B_{k+1}$ is a binomial tree made up of 2 binomial trees $B_k$. Therefore, the total number of nodes in $B_{k+1}$ is the sum of the nodes in both $B_k$ subtrees. By our inductive hypothesis:
    \begin{align*}
        \text{\# nodes} & = 2^k + 2^k,\\
        & = 2(2^k),\\
        & = 2^{k+1}. \quad \checkmark
    \end{align*}
    \item the height of the tree is $k$\\
    \textbf{Base Case:} If $k=0$, height is also zero. $B_0$ is a binomial tree with only 1 node, so its height is 0. \\
    \textbf{Inductive Hypothesis:} For any $k \geq 0$, a binomial tree $B_k$ has height $k$.\\
    \textbf{Induction Step:} Prove that binomial tree $B_{k+1}$ has height $k+1$.\\
    $B_{k+1}$ is a binomial tree made up of 2 binomial trees $B_k$ with the root of one raised as the root of the other. Therefore, the height of $B_{k+1}$ is the height of $B_k + 1$. By our inductive hypothesis:
    \begin{align*}
        \text{height} & = k + (1),\\
        & = k + 1. \quad \checkmark
    \end{align*}
    \item there are exactly ${k \choose i}$ nodes at depth $i$ for $i = 0, 1, . . . , k$\\
    \textbf{Base Case:} If $k=0$, depth can only be $i=0$ (single root node). ${0 \choose 0}$ gives $1$ node. $B_0$ is a binomial tree with only 1 node, so its depth is 0, and the root is its only node. \\
    \textbf{Inductive Hypothesis:} For any $k \geq i$, a binomial tree $B_k$ has ${k \choose i}$ nodes at depth $i$.\\
    \textbf{Induction Step:} Prove that binomial tree $B_{k+1}$ has ${k+1 \choose i}$ nodes.\\
    $B_{k+1}$ is a binomial tree made up of 2 binomial trees $B_k$ with the root of one raised as the root of the other. The tree $B_k$ that became the leftmost child of the now root tree $B_k$ has the same group of nodes at depth$-1$ of $B_{k+1}$ now. Therefore, the number of nodes at depth $i$ is the sum of the number of nodes in $B_k$ at depth $i$ + the number of nodes in $B_k$ at depth $i-1$. By our inductive hypothesis:
    \begin{align*}
        {k + 1 \choose i} = \frac{(k+1)!}{i!(k + 1 - i)!} = {k \choose i} + {k \choose i - 1}.\\
    \end{align*}
    Expand:
    \begin{align*}
        {k \choose i} + {k \choose i - 1} & = \frac{k!}{i!(k - i)!} + \frac{k!}{(i - 1)!(k - i + 1)!},\\
        & = \frac{k!}{i(i - 1)!(k - i)!} + \frac{k!}{(i - 1)!(k - i + 1)(k - i)!},\\
        & = \frac{k!(k - i + 1)}{i(i - 1)!(k - i)!(k - i + 1)} + \frac{k!(i)}{i(i - 1)!(k - i)!(k - i + 1)},\\
        & = \frac{k!(k - i + 1) + k!(i)}{i(i - 1)!(k - i)!(k - i + 1)},\\
        & = \frac{k!(k - i + 1 + i)}{i(i - 1)!(k - i)!(k - i + 1)},\\
        & = \frac{k!(k + 1)}{i(i - 1)!(k - i)!(k - i + 1)},\\
        & = \frac{(k+1)!}{i!(k + 1 - i)!}. \quad \checkmark
    \end{align*}
    \item the root has degree $k$, which is greater than that of any other node; moreover, as the Figure below shows, if we number the children of the root from left to right by $k-1, k-2, . . . , 2, 1, 0$, then child $i$ is the root of a subtree $B_i$\\
    \textit{Proof.} By the binomial tree's recursive definition, $B_k$ is comprised of two binomial trees $B_{k-1}$, and each tree $B_{k-1}$ is comprised of two more trees $B_{k-2}$, and so on. Therefore, by this recursive definition, $B_k$ has higher degree than $B_{k-1}$, $B_{k-1}$ has higher degree than $B_{k-2}$, etc..\\
    \textit{Proof.} By the binomial tree's recursive definition, to create a tree $B_k$, subtrees $B_{k-1}$ are created, then $B_{k-2}$, and so on until $B_0$. The leftmost subtree will then be $B_0$. Ordering the children of the root by $k-1, k-2, . . . , 2, 1, 0$ causes $B_0$ to become the $0$-th child. $B_1$ becomes the $1$-st child, and so on.

\end{enumerate}
% \begin{align}
%     \sin(x) = x, \quad   &\sin(x) = x - \frac{x^3}{3!}, \\
%     \cos(x) & = 1 + \frac{x^2}{2} - \cdots, \\
%     & = 1 \quad\quad \mathrm{for\ x \approx 0}
% \end{align}

% --------------------------------------------------------------
%     You don't have to mess with anything below this line.
% --------------------------------------------------------------
 
\end{document}

